\documentclass{article}

\usepackage{omndoc}

\title{Omndoc: a manual}
\author{Phillip Lord}
\omndoc{omndoc_manual.omn}


\begin{document}
\maketitle

\begin{abstract}
  This document describes the OMNDoc system, which is designed to support
  literate development of Ontologies using the Manchester OWL syntax. At the
  moment, the system is very preliminary!
\end{abstract}

\section{Introduction}
\label{sec:introduction}

An ontology is meant to fulfil two functions: firstly, it is meant to
represent knowledge in an way which is explicit and precise for a human and,
secondly, it provides a computational artifact which is meant to do something.
Unlike a normal computer program, though, the ``something'' in this case is a
varied. In some cases, the computational properties are there solely
to support the development of the human artifact; sometimes, it can provide
more enhanced query or browsing over data described with the ontology;
sometimes, the computational properties are used to predict new knowledge.

As a result, an ontology much more so than a program is much more of a
narrative document. As well as the axiomatisation and essential user
documentation, we wish to document use cases, design decisions, notices for
use and so on. Currently, this is being done in a slighly \textit{ad hoc} way;
usually by adding to the documentation within the ontology. This works to some
extent, but the documentation environment is fairly poor: no sections,
cross-references, bibliography and so on. 

OMNDoc is an attempt to bridge this gap. Instead of writing an Ontology and
then some documentation, the idea is that you write a single source file which
can be used to generate both the documentation and the axiomatisation. This is
an old idea -- literate programming -- which is perhaps more relevant to
ontology building than more traditional programming tasks. 

\section{Usage}
\label{sec:usage}



\end{document}
